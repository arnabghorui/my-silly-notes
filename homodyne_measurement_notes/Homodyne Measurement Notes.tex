\documentclass[english,10pt,a4paper]{article}
\usepackage[T1]{fontenc}
\usepackage{graphicx}
\usepackage{mathtools}
\usepackage{amssymb}
\usepackage{amsthm}
\usepackage{bbm}
\usepackage{babel}
\usepackage[colorlinks=true, allcolors=blue]{hyperref}
\newtheorem{definition}{Definition}

\title{Notes on Homodyne Measurement}
\author{Arnab Ghorui}

% shorthands
\newcommand{\rr}{\hat {\textbf{r}}}
\newcommand{\xx}{\hat{\textbf{x}}}
\newcommand{\dd}[1]{\hat{D}_{#1}}
\newcommand{\ham}{\hat H}
\newcommand{\half}{\frac{1}{2}}
\newcommand{\tr}[1]{\text{Tr}\left[{#1}\right]}
\begin{document}
	\maketitle
	
	\section{Notations}
	\begin{itemize}
		\item $\xx= (\hat x_1, \hat p_1,\cdots,\hat x_n, \hat p_n)^T$, vector of cannonical operators.
		\item $\Omega = \bigoplus_{j=1}^n \Omega_1$, where $\Omega_1 = \begin{pmatrix}
			0 & 1 \\ -1 & 0
		\end{pmatrix}$.\\
		Note that, for $n=1$, $[\hat x_i,\hat x_j] = i[\Omega_1]_{ij}$. Compactly, 
		\begin{equation}
			[\xx, \xx^T] = i\Omega, \tag{Canonical Commutation Relation}
		\end{equation}
		 where, think the commutation relation as element wise commutator.
		\item Borrowing from the optical and field-theoretical terminologies, canonical degrees	of freedom are also referred to as ‘\textit{modes}’.
		\item $\hat a_j = \frac{\hat x_j + \hat p_j}{\sqrt 2}$, annihilation operator.
		\item \textbf{BCH formula: }$e^{A+B}=e^A e^B e^{-\half[A,B]}$	for operators $A,B$ if $[A,[A,B]]=[B,[B,A]]=0$
	\end{itemize}
	
	\section{Prerequisits}
	\subsection{Displacement operators}
	\label{sec:displacement_operators}
	\begin{definition}[Weyl operators]
		\begin{equation}
			\dd{\xi} =e^{i \xi^T \Omega \xx} = e^{i(\hat{x}_1\xi_2-\hat{p}_1\xi_2)}\otimes\cdots\otimes e^{i(\hat{x}_n\xi_{2n}-\hat{p}_n\xi_{2n-1})},
		\end{equation}
		where, $\xi \in \mathbbm{R}^{2n}$.
	\end{definition}
	
	\textbf{Properties:}	\begin{itemize}
		\item $\dd{\xi}^\dagger \dd{\xi} = \mathbbm{1}$ (Unitary operator).
		\item $\dd{\xi}\dd{\xi}=\dd{2\xi}$.
		\item $\dd{\xi}\dd{\eta} = e^{-\frac{i}{2}\xi^T\Omega \eta}\ \dd{\xi+\eta}$. (\textbf{Prove!})
		\item$\dd{-\bar{\xi}} \xx \dd{\bar{\xi}} = \xx - \bar{\xi}$ (\textbf{Prove!})
		\item $\dd{-\bar{\xi}} = \dd{\bar{\xi}}^\dagger$.
	\end{itemize}
	
	
	
	\subsection{Symplectic Group}
	\textbf{TODO:} Linear canonical transformation and Symplectic group, Canonical transformations are those which respect  \textbf{CCR}.
	\begin{definition}
		[Symplectic group]
		\begin{equation}
			S\in Sp_{2n,\mathbbm{R}} \iff S\Omega S^T=\Omega
		\end{equation}
	\end{definition}
	
	\subsection{Normal Modes}
	\textbf{TODO: }Definition, etc.
	
	\section{Gaussian States}
	
	\subsection{Quadratic Hamiltonian and Gaussian States}
	
	The most general quadratic/second-order hamiltonian can be written as follows.
	\begin{equation}
		\label{eq:quad_ham}
		\ham = \half \xx^T H \xx + \xx^T \mathbf{\xi}.
	\end{equation}
	Here, $\mathbf{\xi}$ is a $2n$-dimensional real vector. $H$ is a $2n\times 2n$ symmetric matrix called \textit{Hamiltonian matrix}, not to be confused with Hamiltonian. It can always be taken as  a symmetric matrix because, the antisymmetric part with give a term proportional to identity matrix due to \textbf{CCR}, which can always be discarded. If we take $\bar{\mathbf{\xi}}=H^{-1}\xi$, then $\hat{H}' = \half (\xx-\bar{\xi})^T H (\xx-\bar{\xi})$ is equivalent to $\hat{H}$ up to some additive constant term. Using the fourth property from section \ref{sec:displacement_operators} we can write,
	\begin{align}
		\hat{H}' = \half (\xx-\bar{\xi})^T H (\xx-\bar{\xi}) &= \half (\dd{-\bar{\xi}} \xx \dd{\bar{\xi}})^T H (\dd{-\bar{\xi}} \xx \dd{\bar{\xi}}) \\ &= \half \dd{-\bar{\xi}} \xx^T H \xx \dd{\bar{\xi}}
	\end{align} See Serafini (eq. 3.17) for proof.
	\begin{definition}[Gaussian State]
		Gaussian states are defined as all the ground and thermal states of second-order Hamiltonians [eq.\ref{eq:quad_ham}] with positive definite Hamiltonian matrix H > 0.
	\end{definition}
	Thus a \textit{Gaussian state} can be written as,
	\begin{equation}
		\label{eq:gaussian_state}
		\rho_G = \frac{e^{-\beta \hat{H}}}{\tr{e^{-\beta \hat{H}}}},
	\end{equation}
	where, $\beta >0$ and $\hat{H}$ is defined in Eq. \ref{eq:quad_ham}. Ground state is the limiting value,
	\begin{equation}
		\label{eq:ground_state}
		\rho_G = \lim_{\beta\to \infty} \frac{e^{-\beta \hat{H}}}{\tr{e^{-\beta \hat{H}}}}.
	\end{equation}
	\textbf{Note:}
	\begin{itemize}
		\item All Gaussian states are mixed state by construction, except for the ground state.
		\item Gaussian states are parametrized by $\beta,\ \mathbf{\xi}$ and $H$. Though $\beta$ is redundant and can be absorbed into $H$, it allows one to single out pure Gausian states as a limiting case like in Eq. \ref{eq:ground_state}. 
		\item Gaussian states can be generated First and second moment of quadrature. We'll talk about them later.
	\end{itemize}
	
	\section{Gaussian operations}
	Gaussian operations are CP-maps those take Gaussian states to Gaussian states.
	\subsection{Gaussian Unitaries}
	One may write most general second order hamiltonian of $n$-modes as, $\hat{H}=\half \dd{-\bar{\xi}} \xx^T H \xx \dd{\bar{\xi}} $.
	First note that, $(\dd{-\bar{\xi}} \xx \dd{\bar{\xi}}) (\dd{-\bar{\xi}} \xx \dd{\bar{\xi}}) = \dd{-\bar{\xi}} \xx^2 \dd{\bar{\xi}}$. Then, 
	it is clear that, $ e^{i\hat{H}} =  \dd{-\bar{\xi}} e^{\frac{i}{2}\xx^T H \xx } \dd{\bar{\xi}}$.




\end{document}